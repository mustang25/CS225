\documentclass[11pt, oneside]{article}   	% use "amsart" instead of "article" for AMSLaTeX format
\usepackage{geometry}                		% See geometry.pdf to learn the layout options. There are lots.
\geometry{letterpaper}                   		% ... or a4paper or a5paper or ... 
%\geometry{landscape}                		% Activate for for rotated page geometry
%\usepackage[parfill]{parskip}    		% Activate to begin paragraphs with an empty line rather than an indent
\usepackage{graphicx}				% Use pdf, png, jpg, or eps§ with pdflatex; use eps in DVI mode
								% TeX will automatically convert eps --> pdf in pdflatex		
\usepackage{amssymb}

\title{HW 6.3}
\author{Rob Navarro}
%\date{}							% Activate to display a given date or no date

\begin{document}
\maketitle
%\section{}
%\subsection{}

\noindent
Section 10.5\\
10. Can someone cross all the bridges shown in this map exactly once and return to the starting point?\\
Since the degree of each vertex of the graph is even, the graph has a Euler circuit. Therefore, someone can indeed cross all six bridges exactly once and return to the starting point. \\\\
14. Since the picture is a graph in which there are only two vertices of degree we know it has a Euler path. This means that the picture can be drawn without lifting the pencil if the Euler path is followed. \\\\
26. For which values of n do these graphs have an Euler circuit?\\\\
a) $K_n$\\
The degree of each vertex of $K_n$ is $n - 1$, for any $n\geq 1$. So, the degree of each vertex is even only when $n$ is odd. This means that $K_n$ has a Euler circuit for all $n\geq 3$ when n is odd. \\\\
b) $C_n$\\
$C_n$ has a vertex of two for all $n\geq 3$. So, there is a Euler circuit for all $n\geq 3$. \\\\
c) $W_n$\\
For all $W_n$ where $n\geq 3$ there are more than two vertices of odd degree. Therefore, $W_n$ has no Euler circuit for all $n\geq 3$. 
\end{document}  