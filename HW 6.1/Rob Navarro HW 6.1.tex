\documentclass[11pt, oneside]{article}   	% use "amsart" instead of "article" for AMSLaTeX format
\usepackage{geometry}                		% See geometry.pdf to learn the layout options. There are lots.
\geometry{letterpaper}                   		% ... or a4paper or a5paper or ... 
%\geometry{landscape}                		% Activate for for rotated page geometry
%\usepackage[parfill]{parskip}    		% Activate to begin paragraphs with an empty line rather than an indent
\usepackage{graphicx}				% Use pdf, png, jpg, or eps§ with pdflatex; use eps in DVI mode
								% TeX will automatically convert eps --> pdf in pdflatex		
\usepackage{amssymb}

\title{HW 6.1}
\author{Rob Navarro}
%\date{}							% Activate to display a given date or no date

\begin{document}
\maketitle
%\section{}
%\subsection{}

\noindent 
8. How many different three-letter initials with none of the letters repeated can people have?\\\\
We will have 26, 25, and 24 for options. By the product rule 26 * 25 * 24 = 15600. \\\\
12. How many bit strings are there of length six or less, not counting the empty string?\\\\
Since each bit string can be either 0 or 1 we can find the total by using product rule and sum rule:\\
$2^1 + 2^2 + 2^3 + 2^4 + 2^5 + 2^6 = \frac{2^7 - 1}{1} = 127$\\\\
16. How many strings are there of four lowercase letters that have the letter x in them?\\\\
Let S be the number of possible strings with length less than or equal to 4. We get $S = 26^4$\\
Let T be the number of possible strings without x and less than or equal to 4. We get $T = 25^4$\\
To find the total number of string with x in them we can calculate $S - T = 26^4 - 25^4 = 66351$\\\\
26. How many strings of four decimal digits\\\\
The total number of possible four decimal digits has 10 options for each digit. With this in mind we can use the product rule: $10^4 = 10000$ strings. So 10000 is the amount of possible strings for length 4. \\\\
a) do not contain the same digit twice?\\
If we have no digit repeating the 2nd place will be occupied by a digit in 9 different ways. The 3rd and 4th place will also be represented by a digit in 8 and 7 ways respectively. Using the product rule we get:\\
$10 * 9 * 8 * 7 = 5040$\\\\
b) end with an even digit?\\
The total number of possible digits is 10 and the number of even digits is 5\\
Since the string ends with an even digit we can use the product rule:\\
$10 * 10 * 10 * 5 = 5000$\\\\
c) have exactly three digits that are 9\\
With three positions being occupied by 9 the 4th position can be occupied with any number but 9. This means that the number can be defined 9 different ways 4 times. By the sum rule we get:\\
$9 + 9+ 9 + 9 = 36$\\\\
28. How many license plates can be made using either three digits followed by three uppercase English letters or three uppercase English letters followed by three digits?\\\\
There are 26 options for the 3 letters and 10 choices for numbers. Using the product rule we get:\\
$10 * 10 * 10 * 26 * 26 * 26 * 2 = 35152000$\\\\
48. How many bit strings of length seven either begin with two 0s or end with three 1s?\\\\
We can use the following equation: $|A\cup B| = |A| + |B| - |A - B|$ where:\\
$|A| =$ number of strings of length 7 begins with two 0s = $2^5$\\
$|B| =$ Number of strings end with three 1s = $2^4$\\
$|A\cap B| =$ number of string both beginning with two0s and end with three 1s = $2^2$\\
We then get $2^5 + 2^4 - 2^2 = 44$\\\\
52. Every student in a discrete mathematics class is either a computer science or a mathematics major or is a joint major in these two subjects. How many students are in the class if there are 38 computer science majors (including joint majors), 23 mathematics majors (including joint majors), and 7 joint majors?\\\\
Computer science major = C\\
Mathematics major = M\\
$n(C) = 38, n(M) = 23, n(C\cap M) = 7$\\
$n(C\cup M) = n(C) + n(M) - n(C\cap M) = 38 + 23 - 7 = 54 $\\\\ 
72. Use mathematical induction to prove the product rule for m tasks from the product rule for two tasks.\\\\
Base case: m = 2; the first task can be done in $n_1$ and the second task can be done with $n_2$\\
Then the product rule is just $n_1n_2$, which is the product rule for two tasks. \\
Inductive case: Assume that $P(m)$ is true, we will prove that $P(m + 1)$ is also true.\\\ 
In this case we will get $(n_1n_2...n_m)n_{m+1}$ This proves that by induction $P(m+1)$ is true.\\\\





\end{document}  