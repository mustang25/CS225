\documentclass[11pt, oneside]{article}   	% use "amsart" instead of "article" for AMSLaTeX format
\usepackage{geometry}                		% See geometry.pdf to learn the layout options. There are lots.
\geometry{letterpaper}                   		% ... or a4paper or a5paper or ... 
%\geometry{landscape}                		% Activate for for rotated page geometry
%\usepackage[parfill]{parskip}    		% Activate to begin paragraphs with an empty line rather than an indent
\usepackage{graphicx}				% Use pdf, png, jpg, or eps§ with pdflatex; use eps in DVI mode
								% TeX will automatically convert eps --> pdf in pdflatex		
\usepackage{amssymb}

\title{Quiz 4}
\author{Rob Navarro}
%\date{}							% Activate to display a given date or no date

\begin{document}
\maketitle
%\section{}
%\subsection{}
\noindent
1.Let $A = \{x | 2 < x <  5\}$, $B = \{x | 4\leq x\leq 7\}$ and $C = \{x | 2\leq x < 6\}$, where $x$ represents a real number. Determine the sets \\\\
$A = \{3, 4\}, B = \{4, 5, 6, 7\}, C = \{2, 3, 4, 5\}$\\
$(A - C)\cup A = \{3, 4\}$\\
$(A\cap B) - C = \{\emptyset$\}\\
$B\cap\overline{\rm C} = \{6, 7\}$\\\\
2. For any sets A, B and C , prove that\\
$ A-(B\cap C)\subseteq(A - B)\cup(A - C).$\\\\
Assume: $x\in A - (B\cap C$)\\
then: $x\in A\land (x\notin B\lor x\notin C)$\\
then: $(x\in A\land x\notin B)\lor (x\in A\land x\notin C)$\\
then: $(x\in A - B)\lor (x\in A - C)$\\
then: $x\in (A - B)\cup (A - C)$\\\\
Assume: $x\in (A - B)\cup (A - C)$\\
then: $(x\in A - B)\lor (x\in A - C)$\\
then: $(x\in A\land x\notin B)\lor (x\in A\land x\notin C)$\\
then: $x\in A\land (x\notin B\lor x\notin C)$\\
then: $x\in A - (B\cap C$)\\\\
3. Let A and B are the sets . Use the laws from the following table to show that \\
$\overline{\rm \overline{\rm A}\cup\overline{\rm B} - A} = A$\\
$\overline{\rm \overline{\rm A}\cup\overline{\rm B} - A} $\\
$\equiv (A\cap B) - \overline{\rm A}$\\
$\equiv (A - \overline{\rm A})\cup (B - \overline{\rm A})$\\
$\equiv A$
When subtracting the compliment of A from A we are left with just A. When subtracting the compliment of A from B we are left with only elements that are in A. The union between these two differences will then give us A. \\\\
4. 1. $a_0 = 1, a_1 = -1, a_2 = 8, a_3 = -27$\\
\indent2.  $a_0 = 2, a_1 = 2, a_2 = 2, a_3 = 2$\\\\
5. 1) $\sum\limits_{k=50}^{100} 4k^2 = 4*\sum\limits_{k=50}^{100} k^2 = 4 *\sum\limits_{k=1}^{100} k^2 - \sum\limits_{k=1}^{49} k^2 = 4*(\frac{100 * 101 * 201}{6} - \frac{49 * 50 * 99}{6})$\\
2)$\sum\limits_{i=0}^9 (3^i - 2) = \sum\limits_{i=0}^9 3^i - \sum\limits_{i=0}^9 2 = \frac{3^{10} - 1}{2} - 18$\\
3) $\sum\limits_{i=8}^{10} 6i + 3 = \sum\limits_{i=8}^{10} 6i + \sum\limits_{i=8}^{10} 3 = 6*(\frac{10*11}{2} - \frac{7*8}{2}) - 9$\\
4)$\sum\limits_{i=0}^{10} \frac{1}{2} * -1^i = \frac{1}{2} (\frac{-1^{11} - 1}{-2})$\\



\end{document}  