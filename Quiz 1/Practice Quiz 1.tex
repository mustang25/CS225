\documentclass[11pt, oneside]{article}   	% use "amsart" instead of "article" for AMSLaTeX format
\usepackage{geometry}                		% See geometry.pdf to learn the layout options. There are lots.
\geometry{letterpaper}                   		% ... or a4paper or a5paper or ... 
%\geometry{landscape}                		% Activate for for rotated page geometry
%\usepackage[parfill]{parskip}    		% Activate to begin paragraphs with an empty line rather than an indent
\usepackage{graphicx}				% Use pdf, png, jpg, or eps§ with pdflatex; use eps in DVI mode
								% TeX will automatically convert eps --> pdf in pdflatex		
\usepackage{amssymb}

\title{Practice Quiz 1}
\author{Rob Navarro}
%\date{}							% Activate to display a given date or no date

\begin{document}
\maketitle
%\section{}
%\subsection{}

\noindent 5. Consider the propositions:\\
q: Juan is a computer science major.\\
p: Juan is a math major.\\
Use symbolic connectives to represent the proposition ``Juan is a math major but not a computer science major."\\
Answer: $p\wedge\neg q$\\\\
6. Write each of these propositions in the form ``p if and only if q" in
English.\\\\
a) For you to get an A in this course, it is necessary and sufficient
that you learn how to solve discrete mathematics problems.\\
Answer: You will get an A in the course if and only if you learn how to solve discrete mathematics problems.\\
b) If you read the newspaper every day, you will be informed, and conversely.\\
Answer: You will be informant and conversely if and only if you read the newspaper everyday.\\
c) It rains if it is a weekend day, and it is a weekend day if it rains.\\
Answer: It is a weekend day if and only if it is raining.\\
d) You can see the wizard only if the wizard is not in, and the wizard is not in only if you can see him.\\
Answer: You can see the wizard if and only if the wizard is not in. \\\\
7. State the converse, contrapositive, and inverse of each of these implications.\\\\
1) When I stay up late, it is necessary that I sleep until noon.\\
Converse: When I sleep until noon, it is necessary that I stay up late.\\
Contrapositive: When I don't sleep in until noon, it is necessary that I don't stay up late.\\
Inverse: When I don't stay up late, it is necessary that I don't sleep until noon.\\\\
2)  If it snows tonight, then I will stay at home.\\
Converse: If I stay home, then it snowed tonight.\\
Contrapositive:  If I don't stay home, then it did not snow tonight.\\
Inverse: If it doesn't snow tonight, then I will not stay home.\\\\
8. What is the negation of each of these propositions?\\\\

i) To get tenure as a professor, it is sufficient to be world famous.\\
\indent Answer: p = you are world famous\\
\indent q = get tenure as a professor\\
\indent $p\to q $
\indent Negation: $\neg (p\to q) = p \wedge\neg q$\\
\indent So the answer is, you are famous you and will not get tenure as a professor. \\

ii) If I am lying, I am dying.\\
\indent Answer p = I am lying  q = I am dying\\
\indent $p\to q$
\indent Negation: $\neg (p\to q) = p \wedge\neg q$\\
\indent So the answer is, I am lying and not dying.\\

iii) Tom's smartphone has at least 32GB of memory.\\
\indent Answer: p = the smartphone has at least 32GB of memory
\indent Negation: $\neg p$\\
\indent So the answer is, Tom's smartphone does not have at least 32GB of memory. \\

iv) If the home team does not win, then it is not raining.\\
\indent Answer: p = the home team wins   q = it is raining\\
\indent $\neg p\to\neg q = \neg(\neg p)\lor \neg q = p\lor\neg q$\\
\indent Negation: $\neg (p\lor\neg q) = \neg p\wedge q$\\
\indent So the answer is, The home team does not win and it's raining. \\\\

9. Construct a truth table for the statement $(p\to q) \wedge (p\to r).$\\\\
\indent \begin{tabular} {| l | c | c | c | c || r |}
  \hline
  p & q & r & $p \to q$ & $p\to r$ & $(p \to q)\wedge (p\to r)$ \\ \hline
  T & T & T & T & T & T \\ \hline
  F & T & T & T & T & T\\ \hline
  T & F & T & F & T & F\\ \hline
  T & T & F & T & F & F\\ \hline
  F & F & T & T & T & T\\ \hline
  F & T & F & T & T & T\\ \hline
  T & F & F & F & F & F\\ \hline
  F & F & F & T & T & T\\ \hline
\end{tabular}\\\\

\noindent10. Determine whether $[p\wedge (p\to q)]\to q$ is a tautology.\\\\
$[p\wedge (p\to q)]\to q\\$
$\equiv [p\wedge (\neg p\lor q)]\to q$ (Rule of Implication)\\
$\equiv [(p\wedge\neg p)\lor (p\wedge q)]\to q$ (Distributive) \\
$\equiv [F\lor (p\wedge q)]\to q$ (Negation)\\
$\equiv (p\wedge q)\to q$ (Identity)\\
$\equiv \neg(p\wedge q)\lor q$ (Rule of Implication)\\
$\equiv (\neg p\lor \neg q)\lor q$ (De Morgans)\\
$\equiv \neg p\lor (\neg q\lor q)$ (Associative)\\
$\equiv \neg p\lor T$ (Negation)\\
$\equiv T$ (Domination)





\end{document}  